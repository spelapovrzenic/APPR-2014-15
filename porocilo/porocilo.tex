\documentclass[11pt,a4paper]{article}

\usepackage[slovene]{babel}
\usepackage[utf8x]{inputenc}
\usepackage{graphicx}

\pagestyle{plain}

\begin{document}
\title{Poročilo pri predmetu \\
Analiza podatkov s programom R}
\author{Špela Povrženič}
\maketitle{Voda iz vodovoda}

\section{Izbira teme}
V projektu, bom analizirala podatke, ki se navezujejo na vodo iz javnega vodovoda. Najprej bom analizirala porečja Slovenije, iz kjer pride voda v vodovode. Predstavila bom tudi kakšna je preskrba poslovnih subjektov z vodo po področjih dejavnosti v Sloveniji. Nato pa še porabo vode, dobljene iz javnega vodovoda, v gospodinjstvih na prebivalca.

Podatke za moj projekt sem dobila na spletni strani Statističnega urada Republike Slovenije. Ker lahko na statističnem uradu sam izbiraš v kakšni obliki bi imel podatke in tabele (izbiram lahko med datotekami oblike .html, .csv, .txt, .xls. ... ), sem se odločila, da bom za vsako od mojih 3 tabel uporabila drugačno metodo uvoza le-te v R, torej html., csv. in xls. oblike, na tak način pa bom spoznala vse tehnike in načine, kako jih uvoziti v R.

Povezave do podatkovnih tabel:

1. http://pxweb.stat.si/pxweb/Dialog/varval.asp?ma=2750104S&ti=&path=../Database/Okolje/27_okolje/03_27193_voda/01_27501_javni_vodovod/&lang=2

2. http://pxweb.stat.si/pxweb/Dialog/varval.asp?ma=2750301S&ti=&path=../Database/Okolje/27_okolje/03_27193_voda/03_27503_industrija/&lang=2

3. http://pxweb.stat.si/pxweb/Dialog/varval.asp?ma=3268904S&ti=&path=../Database/Okolje/32_trajnostni_razvoj/10_ravnovesje_skromnost/05_32689_naravni_viri/&lang=2

Moj cilj projekta je, da analiziram iz kje največ dobimo vodo, katere so tiste dejavnosti, ki porabijo največ vode ter katere statistične regije porabijo največ vode v gospodinjstvih na prebivalca. 


\section{Obdelava, uvoz in čiščenje podatkov}

\section{Analiza in vizualizacija podatkov}

\includegraphics{../slike/povprecna_druzina.pdf}

\section{Napredna analiza podatkov}

\includegraphics{../slike/naselja.pdf}

\end{document}

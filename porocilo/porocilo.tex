\documentclass[11pt,a4paper]{article}

\usepackage[slovene]{babel}
\usepackage[utf8x]{inputenc}
\usepackage{graphicx}
\usepackage{hyperref}
\usepackage{pdfpages}
\usepackage{breakurl}
\usepackage{amsmath}

\pagestyle{plain}
\begin{document}
\title{Poročilo pri predmetu \\
Analiza podatkov s programom R\\
\vspace{15mm}
\textbf{\emph{Voda iz javnega vodovoda}}}
\author{Špela Povrženič}
\maketitle


\newpage
\section{Izbira teme}
Za naslov mojega projekta sem izbrala Voda iz javnega vodovoda.
V projektu, bom analizirala podatke, ki se navezujejo na vodo iz javnega vodovoda. 

Uporabila bom pet tabel. 
Najprej bom predstavila porečja Slovenije, iz kjer pride voda v vodovode. Analizirala bom tudi kakšna je preskrba poslovnih subjektov z vodo po področjih dejavnosti v Sloveniji. Nato pa še porabo vode, dobljene iz javnega vodovoda, v gospodinjstvih na prebivalca, torej bom primerjala regije. Uvozila jih bom v \verb|csv| obliki.

\begin{enumerate} 
\item{\url{http://pxweb.stat.si/pxweb/Dialog/varval.asp?ma=2750104S&ti=&path=../Database/Okolje/27_okolje/03_27193_voda/01_27501_javni_vodovod/&lang=2}}

\item{\url{http://pxweb.stat.si/pxweb/Dialog/varval.asp?ma=2750301S&ti=&path=../Database/Okolje/27_okolje/03_27193_voda/03_27503_industrija/&lang=2}}

\item{\url{http://pxweb.stat.si/pxweb/Dialog/varval.asp?ma=3268904S&ti=&path=../Database/Okolje/32_trajnostni_razvoj/10_ravnovesje_skromnost/05_32689_naravni_viri/&lang=2}}

Četrto tabelo sem dobila iz eurostat-a, ki jo bom uvozila v \verb|xlsx| obliki (v Excel tabelo), nato pa jo pretvorila v \verb|csv|. Govori o preskrbi z vodo po področjih dejavnosti v evropskih državah.

\item{\url{http://epp.eurostat.ec.europa.eu/tgm/table.do?tab=table&init=1&language=en&pcode=ten00006&plugin=1}}

Peta tabela pa je iz neke spletne strani, uvozila pa sem jo kot html. Izbrala pa sem jo zato, ker vsebuje urejenostne spremenljivke. Podatki pa so o tem, kolikšna raven arzenika je vsebovana v vodovodnih sistemih v 25 državah.

\item{\url{http://www.nrdc.org/water/drinking/arsenic/chap1.asp}}
\end{enumerate}
Moj cilji projekta so:
\begin{itemize}
\item{Slovenija: da analiziram iz kje največ dobimo vodo, kako se obnaša poraba in preskrba vode čez čas ter katere statistične regije porabijo največ vode v gospodinjstvih na prebivalca.} 
\item{Evropa: Katere države največ porabijo vode ter kakšna bo poraba vode v prihodnosti.}
\end{itemize}
\newpage

\section{Obdelava, uvoz in čiščenje podatkov}

Uporabila sem 5 tabel.
Prve tri sem dobila iz statističnega urada in jih uvozila v \verb|csv| obliki.
Četrto tabelo sem dobila iz eurostat-a, ki sem jo uvozila v \verb|xlsx| obliki (v Excel tabelo), nato pa sem jo pretvorila v \verb|csv|. Peta tabela pa je iz spletne strani, uvozila pa sem jo kot \verb|html|.

Najprej sem tabele, ki sem jih uvozila v \verb|csv| obliki, uredila, se pravi odstranila sem vse odvečne podatke, vrstice. Nato pa sem te tabele uvozila, v mapo \verb|uvoz|, v \verb|uvoz.r|. Pri uvozu sem uporabila tudi \verb|na.strings|, torej tisti podatki, ki niso bili na voljo, sem jih pretvorila v NA(not available). Enako sem naredila tudi četrto tabelo, o evropskih državah.

Pri peti tabeli o arzeniku, ki sem jo dobila na spletni strani, pa sem jo najprej uvozila v mapi \verb|lib|, v \verb|xml.r|. Najprej sem poiskala koliko tabel je v tej spletni strani, nato izbrala tisto, ki sem jo želela uvoziti, potem pa iz nje naredila matriko, nato pa še tabelo. Tudi to tabelo sem nato uvozila v mapi \verb|uvozi|, v \verb|uvozi.r|.

\begin{enumerate}
\item{tabela: Prikazuje količino vode v porečjih Slovenije, od leta 2002 do 2013. Leta [številske spremenljivke] so podana v stolpcih, v vrsticah pa imamo porečja v Sloveniji [imenske spremenljivke].}
\item{tabela: Prikazuje kolikšna je preskrba vode na različnih področjih dejavnosti, med leti 2008 do 2013. Leta [številske spremenljivke] so podana v stolpcu, v vrsticah pa imamo imena različni dejavnosti [imenske spremenljivke].}
\item{tabela: Prikazuje porabo vode, dobljene iz javnega vodovoda, v gospodinjstvih na prebivalca, med leti 2002 do 2012. Leta [številske spremenljivke] so podana v stolpcih, v vrsticah pa so podane regije v Sloveniji [imenske spremenljivke].}
\item{tabela: Prikazuje podatke o preskrbi z vodo po področjih dejavnosti v evropskih državah, med leti 2000 do 2011. Leta [številske spremenljivke] so podana v stolpcih, v vrsticah pa so podane evropske države [imenske spremenljivke].}
\item{tabela: Podatki so o  tem kolikšna raven arzenika je vsebovana v vodovodnih sistemih v 25 državah. V stolpcih so nizke in visoke ocene prizadetih območij, ter nizke in visoke ocene celotnega prebivalsta [imenske spremenljivke], v vrstica pa so podane povprečne stopnje arzenika v ppb [urejenostna spremenljivka].}
\end{enumerate}

Ko sem uvozila vse tabele, sem se lotila risanja grafov. Najprej sem v mapi \verb|slike|, ustvarila novo R-skripto z imenom \verb|grafi.r|, nato napisala v prvi in zadnji vrstici ukaze, ki mi uvozijo grafe v pdf obliko, potem pa sem začela z grafi.\\
\newpage

\textbf{1.GRAF: \emph{Porečja, leto 2013}}: Za prvo tabelo, ki govori o porečjih, sem se odločila da bom uporabila stolpični graf (barplot), saj sem želela, da bi iz grafa videli, koliko je porečij v Sloveniji in katera so največja. Izbrala sem si samo podatke za porečja za leto 2013, saj neke bistvene razlike med leti drugače ni.\\
\textbf{Interpretacija}: Iz grafa lahko vidimo, da Povodje Donave priskrbi daleč največjo količino dobavljene vode, takoj zatem pa porečje Save

\makebox[\textwidth][c]{
\includegraphics[width=1.3\textwidth]{../slike/slike-grafi/porecja.pdf}
}

\newpage
\textbf{2.GRAF: \emph{Preskrba poslovnih subjektov z vodo letno}}: Za drugo tabelo, ki govori o preskrba poslovnih subjektov z vodo, sem izbrala točkast in črtast graf (plot, tipa b), saj sem želela pokazati oskrbo po vseh področjih skupaj, v različnih letih.\\
\textbf{Interpretacija}: Vidimo lahko, da je bila preskrba z vodo največja v letu 2009, nato pa je v dveh letih močno padla in v 2011 dosegla najnižjo raven. Po letu 2011 pa je spet začela preskrba naraščati.\\

\makebox[\textwidth][c]{
\includegraphics[width=1.2\textwidth]{../slike/slike-grafi/preskrba.pdf}
}

\newpage
\textbf{3.GRAF: \emph{Poraba vodovodne vode v gospodinjstvih na prebivalca, leto 2012}}: Za tretjo tabelo, ki govori o poraba vodovodne vode v gospodinjstvih na prebivalca, sem tudi uporabila stolpični graf (barplot), saj sem želela primerjati porabo vode v slovenskih regijah v letu 2012. \\
\textbf{Interpretacija}: Opazimo lahko, da ima Gorenjska, Obalno-kraška, Goriška in Savinjska regija največjo porabo na prebivalca. Najmanjšo porabo pa ima Koroška.\\

\makebox[\textwidth][c]{
\includegraphics[width=1.2\textwidth]{../slike/slike-grafi/regije.pdf}
}

\newpage
\textbf{4.GRAF: \emph{Preskrba z vodo v evropskih državah, leto 2005}}: Tudi za četrti graf, ki govori o preskrbi z vodo v evropskih državah, sem izbrala stolpičnega, saj me zanima, katere so tiste države, ki letno največ porabijo vode. Izbrala sem samo leto 2005, saj je v tej (četrti) tabeli zelo veliko neznanih podatkov, zaradi česar je težko analizirati stvari, zaradi pomanjkanja podatkov, zato sem namenoma izbrala leto 2005, saj je le-to vsebovalo najmanj neznanih podatkov.\\
\textbf{Interpretacija}: Vidimo lahko, da ima Združeno Kraljestvo največjo preskrbo z vodo, nato Španija, Turčija in Francija. Zelo malo pa Slovenija, Danska, Islandija in Latvija.\\

\makebox[\textwidth][c]{
\includegraphics[width=1.4\textwidth]{../slike/slike-grafi/euro.pdf}
}

\newpage
\textbf{5.GRAF: \emph{Količina arzenika v vodovodnih sistemih (število delcev na milijon)}}: Prav tako za peti graf, ki govori o količina arzenika v vodovodnih sistemih, sem izbrala stolpični graf, saj vsebuje urejenostne spremenljivke in je zato najbolj primeren za prikaz podatkov.\\
\textbf{Interpretacija}: Meja, ko je preveč količine arzenika v vodi, in da je potem lahko že zdravlju škodljivo in nevarno, je 10ppb (število delcev na milijardo). Tako iz grafa lahko vidimo, da je veliko količin v normalnem, neškodljivem stanj, se pa najde nekaj malih primerov, v katerih so našli prekomerno količino arzenika v vodi. Največ tistih, ki so že nevarni, so tisti, ki odstopajo za okoli 5ppa od mejne vrednosti.\\

\makebox[\textwidth][c]{
\includegraphics[width=1.3\textwidth]{../slike/slike-grafi/stopnje.pdf}
}


\section{Analiza in vizualizacija podatkov}

V tej fazi sem se najprej odločila katere zemljevide bom vključila v svoj projekt. Glede na to, da obravnavam podatke za Slovenijo in Evropo, sem se odločila za prikaz teh dveh.
\vspace{5mm} 

\subsection{Slovenija}
Prvi zemljevid prikazuje \textbf{povprečno porabo vode na prebivalca}, v letih 2002-2012. Iz zemljevida, po jakosti barve, lahko vidimo, da ima največjo porabo vode Primorska, Gorenjska in Osrednja Slovenija, najmanjšo porabo pa imajo na Koroškem in v Pomurju.

\makebox[\textwidth][c]{
\includegraphics[width=1.5\textwidth]{../slike/slike-zemljevidi/slovenija1.pdf}
}

\newpage
Naslednji zemljevidi prikazujejo \textbf{porabo vode na prebivalca v letih 2003, 2007 in 2012, po regijah v Sloveniji}. Ker sem uporabila spplot, lahko primerjam kako se spreminja poraba vode v letih. Iz teh treh zemljevidov lahko opazimo da v Jugovzhodni Sloveniji poraba na prebivalca z leti narašča, v Osrednjeslovenski, Podravski, Spodnjeposavski, Zasavski in Koroški regiji pa poraba z leti pada. Za ostale regije nemoremo predvidevati dogajanja, saj poraba iz leta v leto niha. 


\makebox[\textwidth][c]{
\includegraphics[width=1.3\textwidth]{../slike/slike-zemljevidi/slovenija2.pdf}
}
\makebox[\textwidth][c]{
\includegraphics[width=1.5\textwidth]{../slike/slike-zemljevidi/slovenija3.pdf}
}
\makebox[\textwidth][c]{
\includegraphics[width=1.5\textwidth]{../slike/slike-zemljevidi/slovenija4.pdf}
}

\newpage
\subsection{Evropa}
Analizirala sem tudi Evropo, namreč \textbf{preskrbo z vodo v Evropi za leto 2005}. Iz zemljevida lahko razberemo, da ima največ preskrbe Združeno Kraljestvo in Španija, ter Turčija in Francija. Zelo malo pa Slovenija, Danska, Islandija in Latvija. Na žalost za ta zemljevid nisem imela podatkov, za vse države, zato je ta analiza malenkost pomankljiva. A vseeno je lepo razvidno, da imajo večje države, večjo preskrbo z vodo, kar je popolnoma logično.


\makebox[\textwidth][c]{
\includegraphics[width=1.45\textwidth]{../slike/slike-zemljevidi/europa.pdf}
}

\newpage

Da pa bi bila analiza evropske preskrbe z vodo bolj natančna in dejansko primerljiva med evropskimi državami, pa sem v naslednjem grafu prikazala \textbf{povprečno preskrbo z vodo v Evropi na prebivalca za leto 2005}. Tako sem še dodatno iz spletne strani \url{http://www.geohive.com/earth/his_proj_europe.aspx} uvozila tabelo v \verb|csv| obliki. Prikazuje število prebivalcev za vsako evropsko državo. V vrsticah so leta 1950, 1960, 1970 ... 2050 [številske spremenljivke], v stolpcih pa so podane imena vseh evropskih držav [imenske spremenljivke]. Nato pa sem podatke iz tabele \verb|euro| normirala z številom prebivalcev v posamezni državi in tako dobila veliko bolj primerljive podatke.\\

Iz zemljevida lahko vidimo, da ima Norveška daleč največjo preskrbo na prebivalca, nato Španija, Irska, Švedska, Združeno Kraljestvo, Portugalska ter Bolgarija in Makedonija. Najmanjšo preskrbo na prebivalca pa ima Litva.
Če pa še omenimo Slovenijo, vidimo da ima v primerjavi z ostalimi, zelo majhno preskrbo z vodo.

\makebox[\textwidth][c]{
\includegraphics[width=1.3\textwidth]{../slike/slike-zemljevidi/europa2.pdf}
}

\newpage

\section{Napredna analiza podatkov}
\subsection{Napredna analiza Slovenije}

Če na kratko povzamem ugotovitve iz prejšnjih faz za Slovenijo:
\begin{enumerate} 
\item{Povodje Donave priskrbi daleč največjo količino dobavljene vode, takoj zatem pa porečje Save}
\item{Preskrba z vodo med leti 2008-2013, je bila največja v letu 2009, nato pa je v dveh letih močno padla in v 2011 dosegla najnižjo raven, po letu 2011 pa je spet začela preskrba naraščati.}
\item{Med leti 2002-2012 ima največjo povprečno porabo vode Primorska, Gorenjska in Osrednja Slovenija, najmanjšo porabo pa imajo na Koroškem in v Pomurju.}
\item{V letu 2012 ima Gorenjska, Obalno-kraška, Goriška in Savinjska regija največjo porabo na prebivalca, najmanjšo pa Koroška.}
\item{V Jugovzhodni Sloveniji poraba vode na prebivalca z leti narašča, v Osrednjeslovenski, Podravski, Spodnjeposavski, Zasavski in Koroški regiji pa poraba z leti pada. Za ostale regije nemoremo predvidevati dogajanja, saj poraba iz leta v leto niha.}
\end{enumerate}

V prejšnjih fazah sem torej ugotovila že mnogo stvari, ki so me na samem začetku zanimale, tako kakšne so razlike porabe vode po regijah ter iz katerih porečij dobimo največ vode. Če pa se ustavim pri \emph{2. ugotovitvi} (torej da je bila preskrba z vodo, 2008-2013, največja v letu 2009, nato je v dveh letih močno padla in v 2011 dosegla minimalno vrednost, nato pa po letu 2011 spet začela naraščati), pa lahko opazim, da ta ugotovitev potrebuje še dodatno analizo, namreč zanima me, kaj je glavni ralog, da je bil v letu 2011 tak velik padec preskrbe vode. Ker je po mojem mnenju količina vode v veliki odvisnosti povezana z padavinami, sem se odočila, da to mojo hipotezo tudi preverim. Uvozila sem novo tabelo iz spletne strani statističnega urada: \url{http://pxweb.stat.si/pxweb/Dialog/varval.asp?ma=0156104S&ti=&path=../Database/Okolje/01_ozemlje_podnebje/10_01561_podnebni_kazalniki/&lang=2}, ki prikazuje \textbf{povprečne letne in mesečne količine padavin letno v Sloveniji, 2008-2013}. Leta [številska spremenljivka] so podana v stolpcih, po vrsticah pa imamo mesečno ter letno povprečno količino padavin [imenska spremnljivka]. \\

Tako sem iz te table, naredila sledeči graf:
\newpage

\textbf{6.GRAF:\emph{Povprečna letna količina padavin v Sloveniji od leta 2008 do 2013}}

\makebox[\textwidth][c]{
\includegraphics[width=1.2\textwidth]{../slike/slike-grafi/padavine.pdf}
}

Že takoj lahko vidimo, da je naš graf o padavinah zelo podoben 2. grafu, grafu o preskrbi, torej iz tega grafa o padavinah lahko razberemo, da je povprečna letna količina padavin v Sloveniji drastično padla leta 2011, potem pa je količina naraščala. Točno tako je bilo tudi pri preskrbi Slovenije z vodo. Torej je preskrba v letu 2011 dosegla minimalno vrednost, zaradi minimalne vrednosti padavin.  Tako sem zgornjo hipotezo potrdila, torej, da je količina vode v veliki odvisnosti povezana z padavinami in tako tudi preskrba Slovenije z vodo. \emph{Večja kot bo povprečna letna količina padavin, večja bo preskrba z vodo}.

\newpage

Ker so vse ugotovitve iz prejšnjih faz popolnoma zaključene, sem se odločila, da bi vseeno lahko še kaj dodatnega analizrala tudi glede količine vode, ki jo dobimo iz porečij in pa kako so ta porečja med seboj povezana, saj sem o porečjih na splošno prej dobila zelo malo ugotovitev.\\

Najprej sem se odločila, da bom iz že prej uporabljene tabele \verb|porecja|, analizirala, koliko vode letno dobimo po porečji ter kako se bo ta količina spreminjala z časom. Najprej sem narisala graf:\\
\textbf{7.GRAF: \emph{Količina vode dobljene po porečjih med leti 2002 do 2013}}

\makebox[\textwidth][c]{
\includegraphics[width=1.4\textwidth]{../slike/slike-analiza/analizaslo2.pdf}
}

Graf pa sem tudi regresijsko analizirala z tremi modeli: \emph{linearni, kvadratni} in \emph{lokalni model loess} (za lokalno prilagajanje). Regresija meri odvisnost dveh slučajnih spremenljivk, torej kakšen vpliv ima ena na drugo.\\
Torej kaj nam te modeli povejo?
\begin{itemize}
\item{\emph{LOKALNI MODEL - loess:} Model za lokalno prilagajanje, se uporablja predvsem za glajenje krivulj. Podano imamo množico točk x in želimo povleči zglajeno krivuljo, ki se bo najbolje prilagajala našim točkam. Na našem grafu lahko vidimo, da je zelena krivulja, taka krivulja, ki se našemu grafu najbolje lokalno prilega.}
\newpage
\item{\emph{LINEARNI MODEL:} Če imamo linearni model, iščemo premico oblike $y = kx + n$. Iz našega grafa in s pomočjo \verb|linearna$coefficients|  lahko vidimo, da je enačba naše modre linearne premice: $y = -1123x + 2423818$ , kjer je x leto in y količina vode celotne Slovenije. Enačbo premice oz. podatka za začetno vrednosti in koeficient, sem dobila s pomočjo spremenljivke \verb|linearna|, ki je nastala med iskanjem linearne zveze. Torej padajoča premica nam pove, da vrednosti količine vode z leti na splošno pada.}

\item{\emph{KVADRATNI MODEL:} Če imamo kvadratni model, iščemo premico oblike $y = a_0 + a_1x +a_2x^2$. Iz našega grafa in s pomočjo \verb|coefficients|  lahko vidimo, da je enačba naše rdeče krivulje, ki se najbolje prilega našemu grafu: $y = 709026300 - 705087.9x + 175.3x^2$. Enačbo krivulje sem dobila s pomočjo spremenljivke \verb|kvadratna|. Iz slike lahko vidimo, da krivulja pada do njene minimalne vrednosti 2011, nato pa začne rasti.}
\end{itemize}

Graf imamo, sedaj pa me zanima, kateri model, linearni ali kvadratni, je tisti, ki se bolje prilega mojim podatkom in je tako bolj točen. Iz grafa lako opazim, da se kvadratna veliko bolj prilega točkam, kot linearna, vendar lahko mojo hipotezo preverim še z izračunom odstopanja napovedanih vrednosti od dejanskih, s pomočjo \verb|vsota.kvadratov|, ki sem jo definirala v \verb|analizaslo.r|. Te vrednosti so preračunane vsote kvadratov razdalj od napovedanih do dejanskih vrednosti. Za linearni model dobimo vrednost 249937183, za kvadratni model pa 208907001. Tisti z manjšo vrednostjo, se mojim podatkom bolje prilage, torej je kvadratni model boljši od linearnega. Če pa pogledamo še loess model, pa se z vrednostjo 74889334, daleč najbolje prilagaja od obeh dveh.

Torej če se sedaj osredotočim na lokalni model loess, ki se najbolje prilagaja, lahko vidim, da so vrednosti od 2002 do 2005 drastično padle, nato so se do leta 2007 malo dvignile, in nato do leta 2012 naprej padale.\\
Mislim, da na ta drastični padec leta 2005 vpliva to, da so v času \textbf{\emph{pred krizo}}, do leta 2007, 2008, na veliko \underline{obnavljali vodovode cevi}, torej so bile manjše izgube na vodovodnem omrežju, tako so bile manjše izgube vode, posledično se je zmanjšala količina vode dobljene iz porečij. Žal to ugotovitev ne morem podkrepiti s še kakšno tabelo, saj ne obstaja nobena statistika, ki bi prikazovala te podatke za celotno Slovenijo, saj je teh podjetij, ki obnavljajo vodovodne sisteme zelo veliko in niso dolžni tega poročati, vsak ima svoje podatke o tem. Če pa že poročajo, pa ponavadi pokažejo evidenco o tem, koliko je bilo stroškov zaradi obnovitve in ne koliko so obnovili po kilometrih.

\textbf{\emph{Po krizi}}, torej po letu 2008 pa vrednosti spet padajo, eden izmed razlogov za to bi bil, da se je v tem času \underline{zmanjšala industrija}, veliko tovarn, obrti, podjetij je propadlo, zaprlo svojo obrt in  posledično se je porabilo manj vode in tako je količina dobljene vode iz porečij padala. Zaradi krize so ljudje postali na splošno veliko bolj \underline{varčni} glede vode, zato se je poraba tudi zmanjšala. Kot primer, že v osnovnih šolah otroke učijo o tem. Tudi \underline{ozaveščenost o okolju} se je povečala in vedno bolj uporabljamo take stroje (npr. pralne, pomivalne), ki so bolj varčni in porabijo manj vode in posledično prihranimo nekaj denarja. Ljudje tudi razmišlajo na bolj \underline{varčen način}, npr. nekateri že uporabljeno vodo v hiši, npr. odpadna voda iz pralnega stroja, potem uporabijo za uporabo v straniščnih školjkah. Torej se ljudje veliko bolj zavedamo, kako pomembna je za nas pitna voda in jo ne porabljamo po nepotrebnem, zato količine vode tudi padajo.\\


Za nasledenjo napredno analizo porečij, sem se odločila, da bi porečja razvrstila v skupine. Za določanje skupin v podatkih sem uporabila metodo razvrščanja v skupine. Izide sem slikovno predstavila z drevesom razvrščanja – dendrogramom.

\makebox[\textwidth][c]{
\includegraphics[width=1.4\textwidth]{../slike/slike-analiza/analizaslo1.pdf}
}

Za hierarhično razvrščanje sem uporabila \verb|hclust| ter Wardovo metodo minimalne variance - \verb|ward.D2|, ki uporablja analizo variance za oceno razdalje med skupinami. Nato sem narisala graf in zatem podatke uvrstila v 6 skupin. Vidimo lahko, da ima povodje Donave največjo vrednost, nato porečje Save. Nato sledi naslednja skupina, ki vsebuje Ljubljansko Savo, porečje Drave in povodje Jadranskega morja, nato pa sledita še dve skupini manjši rek.


\newpage

Za konec analize porečij, pa sem se odločila, da pogledam še naša največja povodja in porečja, ter v kakšnih korelacijah so (korelacija ali korelacijski koeficient je številska mera, ki predstavlja moč linearne povezanosti dveh spremenljivk). Za prikaz sem izbrala metodo \verb|pairs| ter pogledala v kakšnih relacijah so količina celotne vode v Sloveniji, povodje Donave, porečje Drave, Save, Mure ter povodje Jadranskega morja in Soče.

\makebox[\textwidth][c]{
\includegraphics[width=1.4\textwidth]{../slike/slike-analiza/analizaslo3.pdf}
}

\begin{itemize}
\item{\textbf{Slovenija - skupaj}: največjo korelacijo ima z Donavo, Dravo, Savo in Muro. Torej če se količina teh rek poveča, se poveča tudi celotna količina vode iz Slovenije, torej so zelo linearno odvisne.}

\item{\textbf{Povodje Donave}: največjo korelacijo ima z Savo, Dravo in Muro. To je res, saj je po količini vode Sava največji pritok Donave, takoj za njo pa Drava, Mura pa levi pritok Drave, ta pa se, kot smo ravno omenili, zliva v Donavo.}

\item{\textbf{Porečje Drave}: največjo korelacijo ima z Muro, saj se Mura zliva v Dravo, torej sta res linearno odvisni.}

\item{\textbf{Povodje Jadranskega morja}: ima največjo korelacijo z Sočo, kar 0.86, to je tudi pravilno, saj se Soča zliva v povodje Jadranskega morja.}
\end{itemize}

\newpage
\subsection{Napredna analiza Evrope}
Ker sem imela v prejšnjih fazah malo težav pri evropski tabeli, saj je bilo precej držav brez podatkov, je bila analiza zato nenatančna in posledično je bilo zelo težko dejansko primerjati med seboj države, sem se odločila, da uvozim novo tabelo. Uvozila sem jo iz spletne strani \url{http://data.worldbank.org/indicator/ER.H2O.FWTL.K3/countries/1W?display=default}, ki prikazuje \textbf{podatke o količini zajema vode} (to je poraba + izguba vode skupaj), za cel svet. V stolpcih so podana leta 1972, 1977, 1982, 1987, 1992, 1997, 2002, 2007, 2012 in 2013 [številske spremenljivke], v vrsticah pa imamo imena vseh držav sveta [imenske spremnljivke].\\ 
Tabelo sem najprej seveda omejila samo na države Evrope, podatke pa sem nato delila s povprečnim številom prebivalcev vsake države, med leti 1980 in 2020, da lahko dejansko primerjam države med seboj oz. količino zajema vode na prebivalca.\\

Najprej sem se odločila, da države razvrstim v skupine s pomočjo \verb|hclust|. Izbrala sem si leta 2002, 2007 in 2013, saj so le ti vsebovali večina podatkov za vsako državo. Nato sem narisala dendrogram in zatem podatke uvrstila v 7 skupin.

\makebox[\textwidth][c]{
\includegraphics[width=1.4\textwidth]{../slike/slike-analiza/analizaeu1.pdf}
}

\newpage
Odločila sem se, da grupiranje prikažem tudi na zemljevidu, da bo še toliko bolj pregledno, katere države so skupaj.

\makebox[\textwidth][c]{
\includegraphics[width=1.4\textwidth]{../slike/slike-analiza/analizaeu2.pdf}
}

Sedaj ko sem razporedila evropske države po skupinah, me zanima ali za podobne države v isti skupini dobiš podobne modele, oziroma če se razlikujejo med skupinami. Nato bom izbrala tiste najbolj značilne za svojo skupino in naredila graf z modelom zanje.\\

V mapi \verb|slike|, v datoteki \verb|slike-analiza|, sem naredila pdf datoteko z imenom \verb|analizaeu5.pdf|. V njej je 7 grafov, vsak graf za svojo skupino, saj sem želela primerjati vse države z iste skupine na istem grafu, da vidim, kako se vrednosti obnašajo, npr. ali vse rastejo ali padajo ali nič od tega. Zaradi preglednosti in velike količine grafov sem se odločila, da v poročilu prikažem samo prvi graf, graf za prvo skupino, ki vsebuje podatke za 5 držav.

\makebox[\textwidth][c]{
\includegraphics[width=\textwidth]{../slike/slike-analiza/analizaeu5.pdf}
}

Vidimo, da vrednosti v prvi skupini večinoma padajo z časom. Tudi države v drugi in peti skupini so take, da z časom vrednosti padajo. Nakar šesta skupina, ki vsebuje Estonijo in pa sedma skupina, z Slovenijo in Makedonijo, naraščajo. Tretja in četrta skupina pa sta taki, da vrednosti nekaterih držav padajo, nekateri pa naraščajo, torej mešano. Odločila sem se, da v poročilu prikažem samo linearni model za Nemčijo in Slovenijo, saj prvi model pada, drugi pa narašča.

\makebox[\textwidth][c]{
\includegraphics[width=0.5\textwidth]{../slike/slike-analiza/analizaeu-ger.pdf}
\hspace{5mm}
\includegraphics[width=0.5\textwidth]{../slike/slike-analiza/analizaeu-slo.pdf}
}

Torej imamo skupine, ki samo padajo, samo naraščajo ali pa kar oboje. Da pa bi bolje razumelji, kaj se na splošno dogaja z količino vode, sem se odločila, da se za konec analize Evrope osredotočim na to, kako se bo obnašala ta količina vode, za celotno Evropo čez čas, oz. še bolj me zanima, kaj se bo dogajalo z vodo v prihodnosti, npr. do leta 2040.

\newpage
\makebox[\textwidth][c]{
\includegraphics[width=1.3\textwidth]{../slike/slike-analiza/analizaeu4.pdf}
}

Graf sem regresijsko analizirala z dvema modeloma: \emph{linearni} in \emph{kvadratni} model.\\
Torej kaj lahko vidimo iz modelov?
\begin{itemize}
\item{\emph{LINEARNI MODEL:} Če imamo linearni model, iščemo premico oblike $y = kx + n$. Iz našega grafa in s pomočjo \verb|lin$coefficients|, lahko vidimo, da je enačba naše vijolične linearne premice: $y = -1.158x + 2458.791$ , kjer je x leto in y količina zajete vode celotne Evrope. Enačbo premice oz. podatka za začetno vrednosti in koeficient, sem dobila s pomočjo spremenljivke \verb|lin|, ki je nastala med iskanjem linearne zveze. Torej padajoča premica nam pove, da vrednosti količine vode z leti na splošno pada.}

\item{\emph{KVADRATNI MODEL:} Če imamo kvadratni model, iščemo premico oblike $y = a_0 + a_1x +a_2x^2$. Iz našega grafa in s pomočjo \verb|coefficients| lahko vidimo, da je enačba naše rdeče krivulje, ki se najbolje prilega našemu grafu: $y = 402552 - 401.3x + 0.1x^2$. Enačbo krivulje sem dobila s pomočjo spremenljivke \verb|kv|. Iz slike lahko vidimo, da krivulja pada do njene minimalne vrednosti, ki je 2006.5, torej med letoma 2006 in 2007, nato pa začne rasti.}
\end{itemize}

Sedaj pa me zanima, kateri model, linearni ali kvadratni, je tisti, ki se bolje prilega mojim podatkom. Iz grafa lako opazim, da se kvadratna bolj prilega točkam, vendar lahko mojo hipotezo preverim še z izračunom odstopanja napovedanih vrednosti od dejanskih, s pomočjo \verb|vsota.kvadratov2|, ki sem jo definirala v \verb|analizaeu.r|. Te vrednosti so preračunane vsote kvadratov razdalj od napovedanih do dejanskih vrednosti. Za linearni model dobimo vrednost 800.3, za kvadratni model pa 521. Tisti z manjšo vrednostjo, se mojim podatkom bolje prilage, torej je za moj graf kvadratni model bolj točen od linearnega.\\

Torej če sedaj pogledamo ta graf, lahko spet opazimo, da so vrednosti količine vode najprej zelo  padle. Kakor pri Sloveniji, predvidevam, da je na ta padec ogromno vplivalo \underline{obnovitev vodovodnega omrežja}, pred krizo. Države so bile v razcvetu, po celi Evropi se je ogromno gradilo, obnavljalo in tako tudi cevi.  Posledično so bile manjše izgube na vododvodnem omrežju, torej so bile manjše izgube vode, posledično se je zmanjšala količina zajema vode, torej ker je to količina porabe skupaj z izgubo, mislm da je drastično padla predvsem količina izgubljene vode.  Po krizi se je tudi zmanjšala industrija, ljudje so postali bolj ozaveščeni in varčni.\\

Torej vidimo lahko, da se Evropa in Slovenija obnašata zelo podobno, na obe je močno vplivala kriza in pa pred njo, velika obnovitev vodovodnega omrežja. Kaj točno pa se bo dogajalo v prihodosti pa je zaenkrat zelo težko napovedati za obe. Glede na to, da imamo cevi obnovljene ter da smo ljudje vedno bolj ekološki in ozaveščani, bo količina vode še naprej počasi padala kar nekaj let, vendar pa bo spet prišel čas, ko bo cevi treba oboviti, ko bodo začele prepuščati vodo, takrat pa bo odvisno spet v kakšnem fiančnem stanju bo Evropa in Slovenija, ali bo spet kakšna kriza ali pa bo stanje ustaljeno in bomo cevi hitro zamenjali. Glede tega se prepustimo presenetiti, vendar pa mislim, da je za vodo nasplošno lepa prihodnost, saj smo je v veliki meri začeli ceniti in mislim, da bomo lepo poskrbeli zanjo in naše okolje.




\end{document}
